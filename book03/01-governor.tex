\section{The Governor and Indicator}
\begin{mdframed}[backgroundcolor=cyan!5, rightmargin=1em, leftmargin=1em]
The text has been extensively re-organized here while trying to maintain references to the original chapter and line numbers. This  book is a bit of a nightmare to understand, the basic rules are jumbled in with definitions and delineations across two chapters. In reading them I've often referred to Dykes \textsl{Carmen Astrologicum} and \textsl{Persian Nativities Vol. II} as well as Holden's \textsl{Abu'Ali Al-Khayyat: The Judgment of Nativities}, Hand's \textsl{Omar of Tiberias: Three Books on Nativities} and Robert Zoller's DMA Course.
\end{mdframed}

\subsection{Definitions}
Dorotheus describes some of the conditions that effect the ability of a planet to act as either an indicator (hyleg) or a governor (alchocoden):
\begin{description}[style=multiline,leftmargin=5em]
\item[Eastern]\mn{1.4}a planet at least 15°\footnote{The 15° is a generalized value; Dorotheus gives 18° for \Mars, 19° for \Mercury\, and the same is usually assumed (see Dykes) for \Venus. Essentially, you want the planet to be visible, so the distance from the \Sun\, can vary depending on whether the planet is \textsl{superior} (\Saturn, \Jupiter, \Mars) or \textsl{inferior} (\Venus,\Mercury) and also based on the birth latitude. It is sometimes necessary to check actual \href{https://www.sunrise-and-sunset.com/en/sun}{sunrise-sunset times} for the day in question.} from the \Sun\, and rising ahead of it
\item[Western ]\mn{1.5}a planet at least 15° from the \Sun\, and setting after it
\item[USB]\mn{1.6} a planet less than 15° from the \Sun\, is ``under the \Sun's'' rays or beams (usually referenced as USB); it is essentially invisible
\end{description}

\subsection{Indicator and Governor}
The possible indicators (hylegs), the governor (alchocoden), and the conditions they must satisfy. Find the indicator first; begin with the \Sun\, in a day chart, the \Moon\, in a night chart. Once the indicator is identified, by default, its term lord (if it qualifies), becomes the governor.

\subsubsection{The Sun}
In a day chart, the \Sun\, is the default indicator unless it is found to be in one of the following situations:
\vspace{-0.5em}
\begin{itemize}[topsep=0em,itemsep=0em]
\item \mn{1.15}in the 8th or 7th in a feminine sign, look to the \Moon
\item \mn{1.21}if cadent, look to the \Moon\, and if it is also cadent, look to the Ascendant
\end{itemize}

\subsubsection{The Moon}
\vspace{-0.5em}
In a day chart, if the \Sun\, does not qualify as the indicator, the \Moon\, will if she is found in one of the following situations:
\begin{itemize}[topsep=0em,itemsep=0em]
\item \mn{1.17}in the 10th or 11th in a feminine sign
\item \mn{1.18}in the 7th or 8th regardless of sign
\item if the \Moon\, is in the 12th or 9th, regardless of sign, \mn{1.21}use the Ascendant\footnote{The preceding definitions imply that the \Moon\, must be above the horizon in a day chart and not cadent. As the \Sun\, has already been disqualified we must move on to examining the Ascendant.}
\end{itemize}

In a night chart, the \Moon\, is the default indicator unless it is found to be in one of the following situations:
\begin{itemize}[topsep=0em,itemsep=0em]
\item \mn{1.19, 20}the \Moon\, is cadent or its term lord is USB, in which case use the \Sun\, if it is in the 4th or 5th, \mn{1.21}otherwise use the Ascendant
\end{itemize}

\subsubsection{The Ascendant}
If the \Sun\, and \Moon\, have been disqualified from being the indicator, examine the Ascendant degree
\vspace{-0.5em}

\subsubsection{The Term Lord}
\vspace{-0.5em}
\begin{itemize}[topsep=0em, itemsep=0em]
\item \mn{1.18, 19}disqualified if USB
\end{itemize}

