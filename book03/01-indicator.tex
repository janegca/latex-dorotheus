\section{The Indicator (haylij) and Governor (kadhkhudah)}
\begin{mdframed}[backgroundcolor=cyan!5, rightmargin=1em, leftmargin=1em]
The text has been extensively re-organized here while trying to maintain references to the original chapter and line numbers. This  book is a bit of a nightmare to understand, the basic rules are jumbled in with definitions and delineations across two chapters. In reading them I've often referred to Dykes \textsl{Carmen Astrologicum} and \textsl{Persian Nativities Vol. II} as well as Holden's \textsl{Abu'Ali Al-Khayyat: The Judgment of Nativities}, Hand's \textsl{Omar of Tiberias: Three Books on Nativities} and Robert Zoller's DMA Course.
\end{mdframed}

\subsection{Definitions}
Dorotheus describes some of the conditions that effect the ability of a planet to be effective (strong) or ineffective (weak):
\begin{description}[style=multiline,leftmargin=7em]
\item[Eastern]\mn{1.4}a planet at least 15°\footnote{The 15° is a generalized value; Dorotheus gives 18° for \Mars, 19° for \Mercury\, and the same is usually assumed (see Dykes) for \Venus. Essentially, you want the planet to be visible, so the distance from the \Sun\, can vary depending on whether the planet is \textsl{superior} (\Saturn, \Jupiter, \Mars) or \textsl{inferior} (\Venus,\Mercury) and also based on the birth latitude. It is sometimes necessary to check actual \href{https://www.sunrise-and-sunset.com/en/sun}{sunrise-sunset times} for the day in question.} from the \Sun\, and rising ahead of it
\item[Western ]\mn{1.5}a planet at least 15° from the \Sun\, and setting after it
\item[Stationing] either direct or retrograde within 7 days before or after birth (see~appendix \ref{appendix:visibility}:Visibility)
\item[USB]\mn{1.6} a planet less than 15° from the \Sun\, is ``under the \Sun's'' rays or beams (usually referenced as USB); it is essentially invisible
\end{description}

Planets that direct, eastern, and visible are better able to effect what they signify.

\subsection{Possible Indicator}
Find the indicator (hyleg) first; begin with the \Sun\, in a day chart, the \Moon\, in a night chart \textsl{unless} the \Moon\, \mn{1.10} is New, Full, or making a phase, in which case she will be the indicator.

Once the indicator is identified, by default, its term lord (if it qualifies), becomes the governor (alcocoden).

\subsubsection{The Lights}
If\mn{1.24} either light is in its own domicile (\Sun\, in \Leo, \Moon\, in \Cancer) and its term lord is in square or trine that light becomes the indicator and its term lord the governor\footnote{Presumably if this is true for both lights you would need to use the sect light or the best positioned light or possibly, the light whose term lord had the closest aspect.}

\subsubsection{The Planets}
If \mn{1.25} the \Sun\, is in the first degrees \textsl{[0 thru 9?]} of \Aries\, and its term or domicile lord aspects it, that planet becomes the indicator and, presumably, his term lord the governor.

\subsubsection{The Sun}
In a day chart, the \Sun\, is the default indicator unless it is found to be in one of the following situations:
\vspace{-0.5em}
\begin{itemize}[topsep=0em,itemsep=0em]
\item \mn{1.15}in the 8th or 7th in a feminine sign, look to the \Moon
\item \mn{1.21}if cadent, look to the \Moon\, and if it is also cadent, look to the Ascendant
\end{itemize}

\subsubsection{The Moon}
\vspace{-0.5em}
In a day chart, if the \Sun\, does not qualify as the indicator, the \Moon\, will if she is found in one of the following situations:
\begin{itemize}[topsep=0em,itemsep=0em]
\item \mn{1.17}in the 10th or 11th in a feminine sign
\item \mn{1.18}in the 7th or 8th regardless of sign
\item if the \Moon\, is in the 12th or 9th, regardless of sign, \mn{1.21}look to the Ascendant\footnote{The preceding definitions imply that the \Moon\, must be above the horizon in a day chart and not cadent. As the \Sun\, has already been disqualified we must move on to examining the Ascendant.}
\end{itemize}

In a night chart, the \Moon\, is the default indicator unless it is found to be in one of the following situations:
\begin{itemize}[topsep=0em,itemsep=0em]
\item \mn{1.19, 20}the \Moon\, is cadent or its term lord is USB, in which case use the \Sun\, if it is in the 4th or 5th, \mn{1.21}otherwise look to the Ascendant
\end{itemize}

\subsubsection{The Ascendant}
If the \Sun\, or \Moon\, have been disqualified from being the indicator, or their term lords have been disqualified from being the governor, examine the Ascendant degree. If its term lord is disqualified, examine the Ascendant's domicile ruler and \mn{1.12} if it is USB \textsl{[or cadent?]}, examine its lord \textsl{[term lord?]}\footnote{Dorotheus goes on say see it if the place of this ruler is masculine or feminine but that makes little sense in the current context.}

\vspace{-0.5em}
\subsection{Term Lord (Governor)}
If the term lord of the indicator is found to be in one of the following situations  it is disqualified, move on to the next possible indicator.

\begin{itemize}[topsep=0em, itemsep=0em]
\item \mn{1.18, 19} USB
\item \mn{1.11} cadent
\end{itemize}

\subsection{Indications}
\subsubsection{From the Sun}



\subsubsection{From the Moon}
If \mn{1.7} the \Moon\, on the third day after the birth is in the term of:

\begin{itemize}[topsep=0em,itemsep=0em]
\item a benefic which is in a good place trine the \Moon\, while the \Moon\, is in an angular or succedent place ``then say that all of the nativity's condition is good''

\item either \mn{1.9} a benefic or malefic in a good place aspecting the \Moon, ``then it will be mediocre''

\item a \mn{1.8} malefic which is in an angle while Fortune is opposite the \Moon\, ``then say [that there is] no doubt that the nativity is bad''
\end{itemize}

\subsubsection{The Ascendant}
If \mn{1.22,23} the ruler \textsl{[term or domicile? it's not clear]} of the Ascendant is disqualified from being governor ``then say [that there is] ruination in this nativity and that he [the native] will have no upbringing'' unless there is a benefic in the ascendant sign within 15° of the Ascendant.

