\section{Planets in the Turned Chart}
\begin{quote}
\mn{1.185}\textsl{Now I will also make clear to you the transfer of some of the seven to the places of the others.}
\end{quote}

\begin{mdframed}[backgroundcolor=cyan!5, rightmargin=1em, leftmargin=1em]
\textbf{NOTE:} as a rule Dorotheus' instructions in this section of his work are normally read as referring to a \textsl{Solar Return} chart drawn for the date and time the \Sun\, returns to its natal position for the year in question. I am reading them with reference to the \textsl{profected} chart which I take to be the original \textsl{turned} chart.

As the radix chart is turned by 30° each year to bring a new sign to the Ascendant the planets are seen to occupy different houses and so come into new relationships with the radix planets and lots.
\end{mdframed}

Dorotheus \mn{1.186-7} tells us that when a planet reaches the 7th house from its radix position \textsl{``then it is difficult in its maleficence''} and the same is true when it returns to its own radix house. As I understand it, he is saying that if, for example, radix Venus is in the 6th and by profection she appears in the 12th, in opposition to her radix position, then any harm she signifies will be evident in that year and again when she appears back in the 6th, her original place.

In \mn{1.188-9} a diurnal chart, \Mars\, reaching the place of the \Sun\, or \Jupiter\, \textsl{``is worse for this native''} and if a malefic is with the \Sun\, or \Jupiter\, there will be disasters and quarrels with the father In a night chart. In a nocturnal chart, \Saturn\, reaching the place of the \Moon\, is worse and if a malefic is with the \Moon, there is greater harm to his body.

If \mn{1.190} profected \Saturn\, or \Mars\, is trine\footnote{Pingree has this as ``in triplicity'' but that doesn't fit well with the rest of the thought and ``triplicity'' is sometimes confused with ``trine'' in translations (see Dykes p220, although he is referring to Solar Return placements).} the place in the radix, then it will be better for the native than if it is in the 4th (square) or 7th (opposite) place.

If \mn{1.191-2} the malefics are in a right square (i.e. in the 10th from a place), \textsl{``then there will be no good in it''}, but if the benefics are in a right square, \textsl{``then it will be good''}. And if the benefics are trine\footnote{Again, replacing `triplicities' with `trine'.}, \textsl{``then this native will attain good''}.

\begin{quote}
\mn{1.193}\textsl{If the malefics are in their triplicities, then it will harm this native because the lord of the triplicity [being] \Mars\, by day and \Saturn\, by night is the worst of what might be; it is worse for this [native] if the two in the base-nativity were in a place in which [there is] no good.''}
\end{quote}

\begin{mdframed}[backgroundcolor=cyan!5, rightmargin=1em, leftmargin=1em]
The only sense I can make out of the above is that, if the malefics are in difficult houses (2nd, 6th, 8th, 12th) the harm will be worse for the person. The reference to `triplicities' makes little sense at \Mars\, does not rule any triplicity by day and \Saturn\, does not rule any triplicity by night. \Mars\, is the nocturnal triplicity ruler for \Cancer, \Scorpio, and \Pisces, while \Saturn\, is the diurnal triplicity ruler for \Gemini, \Libra, and \Aquarius\, unless he means that if they are in their own triplicities but contrary to the chart sect i.e. \Mars\, in a water sign in a day chart, \Saturn\, in an air sign in a night chart; then they will cause more harm since they are stronger \textsl{in} their own triplicities than out of them.
\end{mdframed}

If \mn{1.94} the person is older than 30 the harm from \Saturn\, will be less.

\begin{quote}
\mn{1.195} \textsl{If \Saturn\, and \Mars\, are rising after the cardines, then also quarrels and misery will reach the native.}
\end{quote}

\begin{mdframed}[backgroundcolor=cyan!5, rightmargin=1em, leftmargin=1em]
For a planet to rise ``after the cardines'', assuming by ``cardines'' he means ``angles'', then it would have to be in the 2nd, 8th, 5th or 11th; only two of which are bad places. Unless he means they are in the 1st, 4th, 7th, or 10th and will, by primary direction, cross the Asc/Desc or MC/IC  after birth. A malefic crossing any angle by primary direction is usually said to be difficult.
\end{mdframed}

If \mn{1.196} \Saturn\, and \Mars, in the radix, are in \textsl{``places in which [there is] no good''} and in the profection the benefics are also brought to places where ``there is no good'', \textsl{``then it will diminish from that good which belonged to the native in that year.''}



