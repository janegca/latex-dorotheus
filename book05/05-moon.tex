\section{The corruption of the Moon}
``Then \mn{1} he mentioned the condition of the \Moon\, and its corruption in which a commencement is not to be made in an action or anything when you find this until the of the \Moon\, and its lord is ameliorated.''

``I \mn{2} will make clear to you its corruption if it is eclipsed, and worse than this if its eclipse is in the sign in which the \Moon\, was at the birth of this native or its eclipse is in trine of the sign in which the \Moon\, was when this native was born.''

``If \mn{3} the \Moon\, is under the \Sun's rays [and] its light is destroyed and it is not seen, then it is corrupted but it is beneficial for one who desires theft or treachery or something which is kept secret against him, and for every hidden or secret action which its master does not wish to be made public.''

``If \mn{4} it is an action which its master wishes to keep secret, commence it when the \Moon\, is immersed under the \Sun's rays as there is good for him, and it will be more concealed if he commences it at the withdrawal of the \Moon\, for the \Sun\, and [its] appearance from under the rays.''

``If \mn{5} the \Moon\, is in the dodecatmoria of \Mars\, or \Saturn\, and if the \Moon\, is in the middle of the equator descending towards the South and if the \Moon\, is in opposition to the \Sun, then it is bad and it indicates the accession of quarrels and that the younger of the two will be the winning antagonist.''

``If \mn{6-7} the \Moon\, is with a malefic or aspecting it and if there is withdrawal of the \Moon\, from the \Sun, in longitude and latitude and if the \Moon\, is in its least motion, that is if it is decreasing in its counting and its motion in a day and a night is less than twelve degrees, then this motion of it is like the motion of \Saturn. If it is thus, then in the action which he commences at that hour will occur difficulty and slowness.''

``If \mn{8} the \Moon\, in its motion is in the path which the learned call ``the burned path'' (the burned path is the middle of the equator, which is \Libra\, and \Scorpio) and if the \Moon\, is in the last degrees of a sign, then it is according to this in the term of \Saturn\, or \Mars, and none of the terms which are at the end of the signs are harder than the terms of these two.''

``If \mn{9} the \Moon\, is cadent toward the ninth from the cardine [which is] the house of government [the tenth] and it happens to be in a double place, then the action which he commences at that hour will be nullified and will not remain in one condition, and a transformation and disturbance will occur in it.''

``Understand \mn{10} when you think about what I wrote for you of the corruption of the \Moon\, and do not commence anything in it as the matter will stop for which it is not fated that it linger on till the condition of the \Moon\, is ameliorated.''

``Let \mn{11-12} \Jupiter\, or \Venus\, be in the ascendent or in the house of government [the tenth]. Look concerning the condition of [these] stars and their positions as, if they happen to be under the [\Sun's] rays or are retrograde in their motion or happen to be in a bad place that is double, which is when it is cadent from a cardine, then they will not have much power.''

\begin{mdframed}[backgroundcolor=cyan!5, rightmargin=1em, leftmargin=1em]
Dykes translation of lines 9, 10, 11 and 12 above is somewhat different; he has:

\textsl{``And if the Moon was falling from the stake of the house of authority towards the ninth, and she occurs in a \textbf{double place}, for the work which he begins in that hour will be nullified and will not remain in one condition, and alteration and unrest will enter into it.''}

\textsl{``So understand when you consider what I have put down for you of the corruption of the Moon, and do not start anything in it; but if a matter comes which your are not able to put off until the condition of the Moon is suitable, then let Jupiter and Venus be in the Ascendant or in the house of authority [10th]''}

\textsl{``And look into the condition of the stars and their positions, for if they fell under the rays or were retrograde in their motion, or they fell into \textbf{a bad place that is double (and it is where is falling from a stake [angle])}, then they will not have much power.''}(p.235) [emphasis added]

Neither Pingree nor Dykes attempt to address the meaning of a ``double place.'' The first few times I read this section I thought that the author, by the term ``double place,'' was referring to a planet being in a mutable (bi-corporeal so double) sign;however, on closer reading I believe he is referring to an angular sign (a cardine) that falls on both an angular and cadent house; a ``double place''. For example a rising (1st) sign with the Asc at 10° would fall on the 1st house and the 12th house as measured when the distance from the MC to the Asc is divided into thirds. Dorotheus appears to be saying that if the planet falls in the cadent section of the sign its effects are ``nullified'' and it ``will not have much power.'' 

This is very similar to the way Valens handles ``destructive and non-destructive rays'' which he describes with reference to operative and non-operative places in Book 3.3.  Valens say a planet ``which projects rays from an angle into inoperative degrees which precede an angle does not become destructive'' (or beneficial). He defines ``inoperative'' areas as the areas cadent from the angles calculated from the MC/IC and Asc/Desc axes.

The key difference is that Valens says the aspect will not be destructive if it falls, for example, in the 12th. He does not say the \textsl{planet} itself is not destructive because it is in the cadent section of the sign. Dorotheus appears to be saying the planet itself will be weak and powerless, not just its rays.
\end{mdframed}

``If \mn{13} any of the malefics is with them in one degree, or they aspect it from trine or quartile or sextile or are with it in one sign, then it indicates that this action will have no strength.''

``If \mn{14} the planets are in signs which the learned call ``the dark''\footnote{Dykes references Sahl in saying ``the dark'' signs are \Libra\, and \Capricorn\, (p.335n23)}, then they will have no power, nor will their rays if they are thus.''\footnote{So possibly Dorotheus is saying both the planets and their rays will have no power if they are in cadent as well as dark places.} 

``Look \mn{15} concerning the totality of every commencement in the manner of Valens the Philosopher\footnote{Dykes believes the author meant ``Dorotheus'' here, not Valens (p.235n25).}; then take into consideration his words as he was making a brilliant [and] learned investigation concerning [these] matters.''

``He \mn{16-17} said: Look concerning the commencement of each action at the \Sun\, and \Moon\, and the lord[s] of the two signs in which the two luminaries are, and together with this look at the ascendent and midheaven. Commence the commencement and action when the \Moon\, is in the ascendent or midheaven or another of the cardines; and [if] the lord of the \Moon's sign happens to be in a double place or cadent\footnote{Here I think he means the planet can be in an angular sign that has a cadent section or in a cadent (i.e. 9th, 12th, 3rd, 6th) sign.}, withdrawing from a cardine, then the beginning of this action will be good but its end bad [and] from this calamity will befall him and whatever he doesn't desire.''

``The \mn{18-19}\Moon\, is the indicator of the base of every action and the lord of the sign in which the \Moon\, is is the indicator of its end. If you find the lord of the \Moon\, in a good place while the \Moon\, is in a bad place, then the commencement of the action is bad but its end is good [and] he will benefit by it, if God---be He exalted!---wishes.''

``If \mn{20} you find the lord of the \Moon\, in a strong place and the \Moon\, in a double place, then it indicates that the beginning of his action will be difficult [and] slow, with no good in it, but [its] end will be good [and] it will come to him as he wishes.''

``If \mn{21} the \Moon\, and its lord happen to be in a strong place, then it indicates that the beginning of that action and its end will happen according to what he desires.''

``If \mn{22} the \Moon\, and its lord happen to be in a double place, then the beginning of this action will be bad and its end worse than the beginning.''

``If \mn{23} the lord of the \Moon\, happens to be in [one of] the signs which follow the cadents\footnote{In a succedent house.}, which are difficult, then it indicates that that action which he commences will have a delay in its beginning and slowness in its end.''

See \mn{24-5} which planet the \Moon\, last separated from and ``[if] you find it flowing from the benefics, then it is good for every action which he commences except for one who wishes to flee from the government or from his land and a fugitive his master as the condition of these is the best of what is [possible] if the flowing of the \Moon\, at that hour is from the malefics, and nothing more suitable is fated for them.''

``Together \mn{26-7} with this look at the lord of the navamsa\footnote{This is the \textsl{novenaria signorum} or 9th-part used by Vedic astrologers. Dykes indicates that they were not used in Hellenistic astrology and that the word was likely inserted by a Persian editor (p237n32).} and at [the planet] which the \Moon\, conjoins with as the consideration concerning the commencement of a matter in this way and its end is not correct until you look concerning its commencement at the lord of the lot of fortune and [concerning] its end at [the planet] with which the \Moon\, conjoins. This section is described in the chapter in which he mentions the matter of flight\footnote{see V.36} and theft\footnote{see V.35}.''

``He \mn{28-9} said: Look at the \Moon\, and the lord of the house in which it is and the star with which the \Moon\, conjoins and how you find the position of the \Moon\, and the star with which it conjoins and the power of these two in the cardines as these two are the indicators of this. If the \Moon\, is with two malefics in one sign and the \Moon\, is immersed between the two in this sign while the benefics aspect it from quartile, then the misfortune and misery in which men suffer at that hour they will escape from and rid of.''

``If \mn{30-1} the \Moon\, happens to be positioned [where] the lords of its triplicity are benefics, and none of the malefics is in opposition to the \Moon, and benefics and malefics aspect it from quartile, then it indicates that they will escape from what happens in it, but they will fall in a second misfortune before they get rid of their first misfortune, [and] then they will escape if God---be He exalted!---wishes. If you find the \Moon\, in trine of the benefics while the malefics aspect it from quartile, then it indicates the like.''

``Look \mn{32} concerning the commencement of every matter at the ascendent and the \Moon. The \Moon\, is the strongest of what is [possible] if it is above the earth, especially if this is at night; the ascendent is the strongest of what is [possible] if the \Moon\, is under the earth by day.''

\begin{quote}
\mn{33-4}\textsl{This is what he says in the introduction to his book, then he writes for each action which people commence a chapter commensurate with it in which [are recorded] the power of the seven and the twelve and on which day and hour the master of that action because of the seven and the twelve must commence each action.}
\end{quote}

\begin{mdframed}[backgroundcolor=cyan!5, rightmargin=1em, leftmargin=1em]
He's basically saying Dorotheus wrote a number of chapters on a variety of activities, offering advice on the time of day and under which planets it would be best to start the activity.
\end{mdframed}








