\chapter{Visibility (Planet Phases)}
\label{appendix:visibility}

A planet's visibility (V) or invisibility (I) depends on its phase/distance from the \Sun:

\begin{table}[h]
\small
\center
\begin{tabular}{l l l}
I & USB & within 15° of the \Sun\\
V & Morning star & superior plants, 15-120°\\
V & 	& \Mercury, 15-28° rising from \Sun \\
V & 	& \Venus, 15-48° rising from \Sun \\
V & Acronychal (\Retrograde) & superior planets, 120-172° \\
I & Curtailed Passage (\Retrograde) & superior planets, 173-187° \\
V &\Retrograde to 2nd Sta. & superior planets, 188-240° \\
V & Evening star & superior planets, 240-345° \\
V &	& \Mercury, 15-28° setting into \Sun \\
V &	& \Venus, 15-48° setting into \Sun \\
\end{tabular}
\caption{Planet Phase (Visibility)}
\end{table}

\noindent \textbf{Lunar Phases}\footnote{References: \textsl{Valens Anthologies} trans. by M. Riley, Ch. 2 \S{35}, \textsl{Ancient Astrology Vol. I} by Demetra George, Ch. 29. } \\
The lunar phases are defined by the \Moon's distance in degrees from the \Sun:
\begin{description}[style=multiline,leftmargin=3em,topsep=0em,itemsep=0em]
\item[0°] New Moon
\item[1°] Rising  (Coming Forth)
\item[45°] Waxing Crescent 
\item[90°] 1st Quarter
\item[135°] 1st Gibbous (Double Convex)
\item[180°] Full Moon
\item[225°] 2nd Gibbous (Disseminating)
\item[270°] 2nd Quarter
\item[315°] Waning Crescent (Balsamic)
\item[345°] Setting
\end{description}