\chapter{Dodecatemoira}
\label{appendix:dodecatemoria}

\textsl{Dodecatemoria} literally means ``the twelfth part''. There are two approaches to finding the dodecatemoria, they both produce the same result by different means\footnote{For reference see the \href{https://www.skyscript.co.uk/gl/dodekatemorion.html}{Skyscript glossary entry for dodecatemorion}, Firmicus' \textsl{Mathesis} Book II.17, James Holden translation or Book II.13 in the Jean Rhys Bram translation;\textsl{Paulus of Alexandria Introduction to Astrology} \S{22}, James Holden translation or the same section in \textsl{Late Classical Astrology: Paulus Alexandrinus and Olympidorus} translated by Dorian Gieseler Greenbaum, MA.}.

Firmicus' method is to multiply the degrees and minutes of the planet  by 12 and then add the full degrees of the position. For example, to find the dodecatemoria of 27 \Aquarius\, 24 we take:
\begin{itemize}
\item[] (27 + 24/60) x 12 = 27.4 x 12 = 328.8
\item[] \Aquarius\, 27°24' = 327.4 + 328.8 = 656.2
\item[] 656.2 - 360 = 296.2
\item[] 296.2 = 296° 12' - 270 = 26 \Capricorn\, 12
\end{itemize}

To find the 12th part of a planet according to Paulus of Alexandria, we  are to  multiply the degrees by 13 and add the degrees of the sign the planet occupies:
\begin{itemize}
\item[] (27 + 24/60) x 13 = 27.4 x 13 = 356.2
\item[] \Aquarius\, begins at 300°
\item[] 356.2 + 300 = 656.2 - 360 = 296.2
\item[] 296.2 = 296° 12' = 26 \Capricorn\, 12
\end{itemize}

The method appears to be based on a Babylonian method that divides each sign by  2.5°, assigning the first 2.5° division to the sign itself, the next 2.5° to the next sign, etc. until you reach the end of the sign. For example, the 12th parts of \Libra\, run 0° \Libra, 2°30' \Scorpio, 5°00' \Sagittarius, 7°30' \Capricorn, etc.  Cuniform tablets have been found describing the multiplication of the degrees by both 12 or 13 but without any explanation (HHA p12n3). 

As shown in the examples above, if you assume the method always accounts for the actual longitude degrees of the planet both methods give the same result;however, those results do not always agree with the Babylonian 2.5° sub-divisions, instead you often end up in the \textsl{next} subdivision. For example, while the above example gives \Capricorn\, as the final sign, the 2.5° subdivision that 27\Aquarius\,24 falls between (22°30' to 25°00') yields \Sagittarius. 

To always arrive at the same sign as the subdivisions, use multiplication by 12 and add 30° x the \textsl{ordinal} number of the sign minus one i.e. \Gemini\, is the 3rd sign so 2 x 30 = 60; \Aquarius\, is the 11th sign so 10 x 30=300, etc.; however this clashes with the actual examples given by Firmicus and Paulus who always cast out 30° for \Aries, neither of them mention the 2.5° subdivision method although dodecatemoria of 2.5° subdivisions do show up in Dorotheus with regards to establishing which degrees of a sign are masculine or feminine (Book I.8).


\section{Examples of using the 12ths}
\subsection{Firmicus}
\begin{quote}
Look also [to see] whether the \Moon, full [of light] by day, throws her dodecatemory into the terms of \Mars, or whether of little light, i.e. waning, throws it into [the terms] of \Saturn, or whether \Mars\, in the DSC...or\Venus\, into [those] of \Mars, and \Mars\, [into those of \Venus... and they take their power from the terms and decans and the sects that may help or that may harm.

The beneficence of \Jupiter\, lapses when his beneficence is afflicted by the weakness of the sign, or the degrees, or the decan, or the exchanging of sect. But also the malevolence of \Saturn\, increases more strongly when, provoked by the quality of the house, or by the terms, or by the decan, or by the sign, or by the [exchanging of] sect, he receives the power to do harm. In a similar fashion [you may judge] the other stars (MH II.17.4-5)
\end{quote}

\subsection{Paulus}
\begin{quote}
The \textsl{d\=odekat\=emorion} of the benefics contributes much whenever it falls in the \textsl{z\=odion} where the \Sun, \Moon, or star of \Mercury\, is, or on one of the four pivots, the Lot of Fortune or Spirit or even Necessity\footnote{The Lot of \Mercury.}, or on the prenatal \textsl{Conjunction} or \textsl{Whole Moon}. For through this theory it will be necessary to distinguish the fortunate, long-lived and blessed. Just as when the \textsl{d\=odekat\=emorion} of the malefic stars falls in the \textsl{z\=odion} where the \Sun, \Moon, or star of \Mercury\, is, or one one of four pivots, or on the Lot of Fortune, Spirit, or Necessity, or on the prenatal \textsl{Conjunction} or \textsl{Whole Moon}, it points out laborers, those unable to acquire property and cursed, since it sets down short lives, violent deaths, sufferings or feebleness (PAG \S22).
\end{quote}

\subsection{Olympidorus}
Olympiodorus says ``That if the \textsl{d\={o}dekat\={e}moria} of the benefic stars are found either on the pivots or on the lots of Fortune, Spirit or Necessity or in the place of the preceding \textsl{Conjunction} or \textsl{Whole Moon} for the nativity, or where the \Sun, \Moon, or \Mercury\, is, in these circumstances the nativity is fortunate and glorious and producing many years for itself.''

He goes on to say the opposite is true if the dodecatemoirion of malefic planets fall in these places ``For it brings about laborers, and those who are ill-spirited and not successful in acquiring property and short-lived, and sometimes violent deaths.'' 
